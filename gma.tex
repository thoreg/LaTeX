%\documentclass[12pt, a4paper, twoside, titlepage]{article}
%\documentclass[12pt, a4paper]{article}
\documentclass[ngerman, 12pt, a4paper]{scrartcl}
%\documentclass[ngerman, 12pt, a4paper]{article}
\usepackage[ngerman]{babel}
\usepackage[T1]{fontenc}
\usepackage[utf8]{inputenc}
% setzt Leerzeichen, wenn welche hingehören (vor nächstem wort, aber nicht vor "." oder ")")
\usepackage{xspace}

%\renewcommand{\familydefault}{\sfdefault}
% define the font for the whole document
\usepackage{times}
% allow inclusion of pdf documents
\usepackage{pdfpages}
% allow inclusion of graphics
\usepackage{graphicx}
\DeclareGraphicsExtensions{.pdf,.png,.jpg}
%
% expliziete Angaben für Silbentrennung
%\hypenation{in-te-grie-ren}
\hyphenation{in-te-grier-ter}
\hyphenation{Pflanz-ge-f\"a\ss}
\hyphenation{Pflanz-ge-f\"a\ss-es}
\hyphenation{wie-de-rum}
\hyphenation{wo-bei}


% font size could be 10pt (default), 11pt or 12 pt
% paper size coulde be letterpaper (default), legalpaper, executivepaper,
% a4paper, a5paper or b5paper
% side coulde be oneside (default) or twoside
% columns coulde be onecolumn (default) or twocolumn
% graphics coulde be final (default) or draft
%
% titlepage coulde be notitlepage (default) or titlepage which
% makes an extra page for title
%
% paper alignment coulde be portrait (default) or landscape
%
% equations coulde be
%   default number of the equation on the rigth and equation centered
%   leqno number on the left and equation centered
%   fleqn number on the rigth and  equation on the left side
%
\title{Gebrauchsmuster Anmeldung - Pflanzgefäß mit integrierter LED-Lichterkette}
\author{Thomas Rega}

\date{\today}
% \date{\today} date coulde be today
% \date{15.12.2012}
% \date{ } or there is no date

\parindent 0pt

\begin{document}
% Hint: \title{what ever}, \author{who care} and \date{when ever} could stand
% before or after the \begin{document} command
% BUT the \maketitle command MUST come AFTER the \begin{document} command!
\maketitle

%\begin{abstract}
%Short introduction to subject of the paper \ldots
%\end{abstract}

%\tableofcontents % create a table of contens

\section*{Beschreibung}
Ein Blumentopf üblicher Bauart ist ein Gefäß zur Aufnahme von Pflanzen samt Erde. Es handelt
sich um einen nach oben offenen Hohlkörper. In der Regel besteht ein Blumentopf aus einem
äußeren Übertopf und einem Innentopf. Materialien für den äußeren Übertopf sind beispielsweise Ton,
Terracotta, Porzellan, Glas, Holz oder Keramik. Der Innentopf ist aus Kunststoff (Polyvinylchlorid)
gefertigt.  Ein Blumentopf ist in der Regel zwischen 10 und 50 cm hoch, wobei es auch sehr kleine
(Höhe \textless 10cm) und sehr große Exemplare (Höhe \textgreater 50cm) gibt.
Blumentöpfe gibt es in verschiedenen Ausführungen: rund, eckig, oval. Im Folgenden wird aus Gründen
der Einfachheit die Bezeichnung "Pflanzgefäß" verwendet, wobei alle Blumentöpfe üblicher Bauart
gemeint sind, in denen eine Einpflanzung möglich wäre. Über beleuchtende Elemente verfügt ein
Pflanzgefäß üblicher Bauart nicht.\newline


Das Gebrauchsmuster ``Blumentopf mit LED-Beleuchtung'' (Aktenzeichen: 20 2011 107 549.8) ist bekannt.
Die im folgenden beschriebene Erfindung grenzt sich von diesem Gebrauchsmuster klar ab. Das
Gebrauchsmuster ``Blumentopf mit LED-Beleuchtung'' bezieht sich auf LED-Strahler, welche am oberen
Rand des Topfes eingelassen sind und nach oben zeigen, um die Pflanze zu beleuchten.
Die hier beschriebene Erfindung ist ein Pflanzgefäß mit integrierter LED-Lichterkette. Die
Stromversorgung der Lichterkette kann auf zwei verschiedene Arten erfolgen. Zum einen kann die
Lichterkette per Akkumulator betrieben werden, welcher wiederum über ein Solar Panel aufgeladen
wird. Zum anderen kann die Stromversorgung der Lichterkette per 230 Volt Spannungsversorgung
realisiert werden. In den Außentopf des Pflanzgefäßes wird entsprechend der Anzahl der LEDs der
Lichterkette für jede LED ein Loch gebohrt. Jedes Loch nimmt eine LED der Lichterkette auf. Die
LEDs sind alle nach außen ausgerichtet - niemals nach oben - und erleuchten so das direkte Umfeld
des Topfes und niemals die sich darin befindende Pflanze. In den Außentopf mit integrierter LED-
Lichterkette wird ein Innentopf eingesetzt, dies ermöglicht die herkömmliche Nutzung des Pflanzgefäßes.\newline

Der Anmeldegegenstand ``Pflanzgefäß mit integrierter LED-Lichterkette'' setzt sich folglich aus
einem Außentopf, einer oder mehrerer LED-Lichterketten mit einer Spannungsversorgung (Solar/230V)
sowie einem Innentopf zusammen. Folgende Komponenten können auf Wunsch des Kunden angepasst werden:
\begin{itemize}
\item Anzahl der verwendeten LEDs (10 < 1000)
\item Farbe der LEDs (z.B.: rot, rosa, weiß, kaltweiß, warmweiß, blau, grün, mehrfarbig)
\item Material des Topfes (Stein, Porzellan, Holz, Ton, Glas, Keramik, Kunststoff)
\item Form/Größe des Topfes (Durchmesser von 10 bis 200 cm)
\end{itemize}

Die Spannungsversorgung übernimmt je nach Ausführung:
\begin{itemize}
\item \textbf{Akkumulator (AA)} - integriert in ein Solarpanel, welches das Ende der
LED-Lichterkette bildet. Durch ein Loch nahe dem Boden des Topfes ist ein Kabel nach außen geführt,
so kann das Solarpanel etwas abseits des Topfes platziert werden
\item \textbf{Steckernetzteil (230 Volt)} -  mit Schalter und Zuleitung
\end{itemize}


Eingesetzt wird die in Schutzanspruch 1 angegebene Erfindung, sowohl zur direkten als auch zur
indirekten Beleuchtung im Außen- und Innenbereich. Die Erfindung lässt sich zur indirekten
Beleuchtung verschiedenster Räumlichkeiten und Freiflächen nutzen. Denkbar wäre der Einsatz im
Außenbereich für gewerbliche Zwecke: in einer Strandbar und/oder Biergarten, im Eingangsbereich
eines Hotels oder eines Restaurants, aber auch im privaten Bereich als Beleuchtung für den
Eingangs-/Einfahrtsbereich, den Garten, den Balkon, die Terrasse oder Veranda. Aber auch im
Innenbereich lassen sich die Pflanzgefäße hervorragend als gestalterisches Mittel einsetzen.\newline

Ein Ausführungsbeispiel der Erfindung wird anhand von Abbildung \ref{fig:schnitt} auf Seite
\pageref{fig:schnitt} erläutert. Abbildung \ref{fig:schnitt} zeigt:
\begin{description}
\item[1]
Außentopf

\item[2]
LED (Lumineszenz-Diode)

\item[3]
Innentopf

\item[4]
Drainage
\end{description}

\begin{figure}[]
\centering
\includegraphics[scale=0.6]{schnitt}
\caption{Querschnitt}
\label{fig:schnitt}
\end{figure}

\newpage
\goodbreak
\pagebreak


\section*{Schutzansprüche}
\begin{enumerate}
\item Pflanzgefäß mit LED-Lichterkette dadurch gekennzeichnet, dass in die Außenwand des
Topfes LEDs eingelassen sind, deren Lichtkegel nach außen gerichtet sind.
\item Pflanzgefäß nach Anspruch 1 dadurch gekennzeichnet, dass die Beleuchtungseinrichtungen
als LED-Lichterkette(n) ausgeführt ist/sind.
\item Pflanzgefäß nach einem der vorhergehenden Ansprüche, dadurch gekennzeichnet, dass der
nahe Außenbereich des Pflanzgefäß angestrahlt wird.
\item Pflanzgefäß nach einem der vorhergehenden Ansprüche, dadurch gekennzeichnet, dass
mindestens zehn, maximal eintausend LEDs in die Außenwand des Topfes eingelassen sind.
\item Pflanzgefäß nach einem der vorhergehenden Ansprüche dadurch gekennzeichnet, dass die
Form des Pflanzgefäß zylindrisch, konisch-zylindrisch, eckig, eckig mit gerundeten Ecken,
polygonförmig oder rechteckig in Form eines Pflanzgefäßes ausgeführt ist.
\item Pflanzgefäß nach einem der vorhergehenden Ansprüche dadurch gekennzeichnet, dass der
Aussentopf des Pflanzgefäßes aus Porzellan, Keramik, Ton, Holz, Wasserlilien, Schilf, Bast, Metall,
Kunststoff oder Glas bestehen kann
\item Pflanzgefäß nach einem der vorhergehenden Ansprüche dadurch gekennzeichnet, dass die
Farbe der LEDs blau, grün, rot, weiß, kaltweiß, warmweiß, rot, rosa, violett oder gelb sein
kann. Die LEDs können auch als Multi-Color LED umgesetzt sein (Farbe der LEDs kann wechseln).
\end{enumerate}

\newpage
\goodbreak
\pagebreak

\section*{Ansichten}
Auf der folgenden Seite ist die Erfindung von allen Seiten und in der Perspektive dargestellt.
In Schnitt A-A sind folgende Elemente durch Zahlen gekennzeichnet:
\begin{description}
\item[1] Außentopf
\item[2] LED (Lumineszenz-Diode)
\item[3] Innentopf
\item[4] Drainage
\end{description}


\includepdf[lastpage=1]{schema.pdf}

\end{document}
